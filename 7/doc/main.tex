\documentclass[a4paper]{article}

\input{style/ch_xelatex.tex}
\input{style/scala.tex}

%代码段设置
\lstset{numbers=left,
basicstyle=\tiny,
numberstyle=\tiny,
keywordstyle=\color{blue!70},
commentstyle=\color{red!50!green!50!blue!50},
frame=single, rulesepcolor=\color{red!20!green!20!blue!20},
escapeinside=``
}

\graphicspath{ {figures/} }
\usepackage{ctex}
\setCJKmainfont[ItalicFont=Noto Sans CJK SC Bold, BoldFont=Noto Serif CJK SC Black]{Noto Serif CJK SC}
\usepackage{graphicx}
\usepackage{color,framed}%文本框
\usepackage{listings}
\usepackage{caption}
\usepackage{amssymb}
\usepackage{enumerate}
\usepackage{xcolor}
\usepackage{bm} 
\usepackage{lastpage}%获得总页数
\usepackage{fancyhdr}
\usepackage{tabularx}  
\usepackage{geometry}
\usepackage{minted}
\usepackage{graphics}
\usepackage{subfigure}
\usepackage{float}
\usepackage{pdfpages}
\usepackage{pgfplots}
\pgfplotsset{width=10cm,compat=1.9}
\usepackage{multirow}
\usepackage{footnote}
\usepackage{booktabs}

%-----------------------伪代码------------------
\usepackage{algorithm}  
\usepackage{algorithmicx}  
\usepackage{algpseudocode}  
\floatname{algorithm}{Algorithm}  
\renewcommand{\algorithmicrequire}{\textbf{Input:}}  
\renewcommand{\algorithmicensure}{\textbf{Output:}} 
\usepackage{lipsum}  
\makeatletter
\newenvironment{breakablealgorithm}
  {% \begin{breakablealgorithm}
  \begin{center}
     \refstepcounter{algorithm}% New algorithm
     \hrule height.8pt depth0pt \kern2pt% \@fs@pre for \@fs@ruled
     \renewcommand{\caption}[2][\relax]{% Make a new \caption
      {\raggedright\textbf{\ALG@name~\thealgorithm} ##2\par}%
      \ifx\relax##1\relax % #1 is \relax
         \addcontentsline{loa}{algorithm}{\protect\numberline{\thealgorithm}##2}%
      \else % #1 is not \relax
         \addcontentsline{loa}{algorithm}{\protect\numberline{\thealgorithm}##1}%
      \fi
      \kern2pt\hrule\kern2pt
     }
  }{% \end{breakablealgorithm}
     \kern2pt\hrule\relax% \@fs@post for \@fs@ruled
  \end{center}
  }
\makeatother
%------------------------代码-------------------
\usepackage{xcolor} 
\usepackage{listings} 
\lstset{ 
breaklines,%自动换行
basicstyle=\small,
escapeinside=``,
keywordstyle=\color{ blue!70} \bfseries,
commentstyle=\color{red!50!green!50!blue!50},% 
stringstyle=\ttfamily,% 
extendedchars=false,% 
linewidth=\textwidth,% 
numbers=left,% 
numberstyle=\tiny \color{blue!50},% 
frame=trbl% 
rulesepcolor= \color{ red!20!green!20!blue!20} 
}

%-------------------------页面边距--------------
\geometry{a4paper,left=2.3cm,right=2.3cm,top=2.7cm,bottom=2.7cm}
%-------------------------页眉页脚--------------
\usepackage{fancyhdr}
\pagestyle{fancy}
\lhead{\kaishu \leftmark}
% \chead{}
\rhead{\kaishu 并行程序设计实验报告}%加粗\bfseries 
\lfoot{}
\cfoot{\thepage}
\rfoot{}
\renewcommand{\headrulewidth}{0.1pt}  
\renewcommand{\footrulewidth}{0pt}%去掉横线
\newcommand{\HRule}{\rule{\linewidth}{0.5mm}}%标题横线
\newcommand{\HRulegrossa}{\rule{\linewidth}{1.2mm}}
\setlength{\textfloatsep}{10mm}%设置图片的前后间距
%--------------------文档内容--------------------

\begin{document}
\renewcommand{\contentsname}{目\ 录}
\renewcommand{\appendixname}{附录}
\renewcommand{\appendixpagename}{附录}
\renewcommand{\refname}{参考文献}
\renewcommand{\figurename}{图}
\renewcommand{\tablename}{表}
\renewcommand{\today}{\number\year 年 \number\month 月 \number\day 日}

%-------------------------封面----------------
\begin{titlepage}
  \begin{center}
    \includegraphics[width=0.8\textwidth]{NKU.png}\\[1cm]
    \vspace{20mm}
    \textbf{\huge\textbf{\kaishu{计算机学院}}}\\[0.5cm]
    \textbf{\huge{\kaishu{并行程序设计第 7 次作业}}}\\[2.3cm]
    \textbf{\Huge\textbf{\kaishu{高斯消去法的 MPI 并行化}}}

    \vspace{\fill}

    % \textbf{\Large \textbf{并行程序设计期末实验报告}}\\[0.8cm]
    % \HRule \\[0.9cm]
    % \HRule \\[2.0cm]
    \centering
    \textsc{\LARGE \kaishu{姓名\ :\ 丁屹}}\\[0.5cm]
    \textsc{\LARGE \kaishu{学号\ :\ 2013280}}\\[0.5cm]
    \textsc{\LARGE \kaishu{专业\ :\ 计算机科学与技术}}\\[0.5cm]

    \vfill
    {\Large \today}
  \end{center}
\end{titlepage}

\renewcommand {\thefigure}{\thesection{}.\arabic{figure}}%图片按章标号
\renewcommand{\figurename}{图}
\renewcommand{\contentsname}{目录}
\cfoot{\thepage\ of \pageref{LastPage}}%当前页 of 总页数


% 生成目录
\clearpage
\tableofcontents
\newpage

\section{问题描述}
高斯消去的计算模式如图 \ref{gauss} 所示,在第$k$步时,对第$k$行从$(k, k)$开始进行除法操作,并且将后续的$k + 1$至$N$行进行减去第$k$行的操作,串行算法如下面伪代码所示。
\begin{figure}
  \centering
  \includegraphics[width=1\textwidth]{gauss.png}
  \caption{高斯消去法示意图}
  \label{gauss}
\end{figure}
\begin{breakablealgorithm}
  \caption{普通高斯消元算法伪代码}
  \begin{algorithmic}[1] %每行显示行号  
    \Function {LU}{}
    \For {$k:=0$\ \textbf{to}\ $n$}
    \For {$j:=k+1$\ \textbf{to}\ $n$}
    \State {$A[k,j]:=A[k,j]/A[k,k]$}
    \EndFor
    \State{$A[k,k]:=1.0$}
    \For {$i:=k+1$\ \textbf{to}\ $n$}
    \For {$j:=k+1$\ \textbf{to}\ $n$}
    \State {$A[i,j]:=A[i,j]-A[i,k]*A[k,j]$}
    \EndFor
    \State{$A[i,k]:=0$}
    \EndFor
    \EndFor
    \EndFunction
  \end{algorithmic}
\end{breakablealgorithm}

观察高斯消去算法,注意到伪代码第 4,5 行第一个内嵌循环中的$A[k, j] := A[k, j]/A[k, k]$以及伪代码第$8,9,10$行双层$for$循环中的$A[i, j] := A[i, j] − A[i, k]×A[k, j]$都是可以进行向量化的循环。可以通过 MPI 对这两步进行并行优化。

\section{算法设计}

源码链接:
\url{https://github.com/ArcanusNEO/Parallel-Programming/tree/master/7}

\subsection{测试用例的确定}

由于测试数据集较大,不便于各个平台同步,所以采用固定随机数种子为 $12345687$ 的 mt19937随机数生成器。
经过实验发现不同规模下,所有元素独立生成,限制大小在 $[0, 100]$,能够生成可以被正确消元的矩阵。

代码如下:

\begin{lstlisting}[frame=trbl,language={C++}]
  uniform_real_distribution<float> dist(0, 100);
  mt19937       mt(12345687);
  int           n;
  istringstream iss(argv[1]);
  iss >> n;
  cout << n << endl;
  for (int i = 1; i <= n; ++i)
    for (int j = 1; j <= n; ++j) cout << dist(mt) << " \n"[j == n];
\end{lstlisting}

\subsection{实验环境和相关配置}

实验在本地 Arch Linux x86\_64 平台和华为鲲鹏服务器平台完成,使用 cmake 构建项目,均开启 O3 加速;

\subsection{默认平凡算法}

使用一维数组模拟矩阵,避免改变矩阵大小时第二维不方便调整、必须设成最大值的问题,可以减少 cache 失效;

使用 $\#define\ matrix(i, j)\ arr[(i) * n + (j)]$ 宏,增强可读性;

\begin{lstlisting}[frame=trbl,language={C++}]
void func(int& ans, float arr[], int n) {
  for (int k = 0; k < n; ++k) {
    for (int j = k + 1; j < n; ++j) matrix(k, j) = matrix(k, j) / matrix(k, k);
    matrix(k, k) = 1.0;
    for (int i = k + 1; i < n; ++i) {
      for (int j = k + 1; j < n; ++j)
        matrix(i, j) = matrix(i, j) - matrix(i, k) * matrix(k, j);
      matrix(i, k) = 0;
    }
  }
}
\end{lstlisting}

% \newpage

% \section{参考文献}
% \cite{1}\cite{2}\cite{3}\cite{4}\cite{5}\cite{6}

% \bibliographystyle{plain}
% \bibliography{Parallel-Programming-0.bib}

\end{document}