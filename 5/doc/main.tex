\documentclass[a4paper]{article}

\input{style/ch_xelatex.tex}
\input{style/scala.tex}

%代码段设置
\lstset{numbers=left,
basicstyle=\tiny,
numberstyle=\tiny,
keywordstyle=\color{blue!70},
commentstyle=\color{red!50!green!50!blue!50},
frame=single, rulesepcolor=\color{red!20!green!20!blue!20},
escapeinside=``
}

\graphicspath{ {figures/} }
\usepackage{ctex}
\setCJKmainfont[ItalicFont=Noto Sans CJK SC Bold, BoldFont=Noto Serif CJK SC Black]{Noto Serif CJK SC}
\usepackage{graphicx}
\usepackage{color,framed}%文本框
\usepackage{listings}
\usepackage{caption}
\usepackage{amssymb}
\usepackage{enumerate}
\usepackage{xcolor}
\usepackage{bm} 
\usepackage{lastpage}%获得总页数
\usepackage{fancyhdr}
\usepackage{tabularx}  
\usepackage{geometry}
\usepackage{minted}
\usepackage{graphics}
\usepackage{subfigure}
\usepackage{float}
\usepackage{pdfpages}
\usepackage{pgfplots}
\pgfplotsset{width=10cm,compat=1.9}
\usepackage{multirow}
\usepackage{footnote}
\usepackage{booktabs}

%-----------------------伪代码------------------
\usepackage{algorithm}  
\usepackage{algorithmicx}  
\usepackage{algpseudocode}  
\floatname{algorithm}{Algorithm}  
\renewcommand{\algorithmicrequire}{\textbf{Input:}}  
\renewcommand{\algorithmicensure}{\textbf{Output:}} 
\usepackage{lipsum}  
\makeatletter
\newenvironment{breakablealgorithm}
  {% \begin{breakablealgorithm}
  \begin{center}
     \refstepcounter{algorithm}% New algorithm
     \hrule height.8pt depth0pt \kern2pt% \@fs@pre for \@fs@ruled
     \renewcommand{\caption}[2][\relax]{% Make a new \caption
      {\raggedright\textbf{\ALG@name~\thealgorithm} ##2\par}%
      \ifx\relax##1\relax % #1 is \relax
         \addcontentsline{loa}{algorithm}{\protect\numberline{\thealgorithm}##2}%
      \else % #1 is not \relax
         \addcontentsline{loa}{algorithm}{\protect\numberline{\thealgorithm}##1}%
      \fi
      \kern2pt\hrule\kern2pt
     }
  }{% \end{breakablealgorithm}
     \kern2pt\hrule\relax% \@fs@post for \@fs@ruled
  \end{center}
  }
\makeatother
%------------------------代码-------------------
\usepackage{xcolor} 
\usepackage{listings} 
\lstset{ 
breaklines,%自动换行
basicstyle=\small,
escapeinside=``,
keywordstyle=\color{ blue!70} \bfseries,
commentstyle=\color{red!50!green!50!blue!50},% 
stringstyle=\ttfamily,% 
extendedchars=false,% 
linewidth=\textwidth,% 
numbers=left,% 
numberstyle=\tiny \color{blue!50},% 
frame=trbl% 
rulesepcolor= \color{ red!20!green!20!blue!20} 
}

%-------------------------页面边距--------------
\geometry{a4paper,left=2.3cm,right=2.3cm,top=2.7cm,bottom=2.7cm}
%-------------------------页眉页脚--------------
\usepackage{fancyhdr}
\pagestyle{fancy}
\lhead{\kaishu \leftmark}
% \chead{}
\rhead{\kaishu 并行程序设计实验报告}%加粗\bfseries 
\lfoot{}
\cfoot{\thepage}
\rfoot{}
\renewcommand{\headrulewidth}{0.1pt}  
\renewcommand{\footrulewidth}{0pt}%去掉横线
\newcommand{\HRule}{\rule{\linewidth}{0.5mm}}%标题横线
\newcommand{\HRulegrossa}{\rule{\linewidth}{1.2mm}}
\setlength{\textfloatsep}{10mm}%设置图片的前后间距
%--------------------文档内容--------------------

\begin{document}
\renewcommand{\contentsname}{目\ 录}
\renewcommand{\appendixname}{附录}
\renewcommand{\appendixpagename}{附录}
\renewcommand{\refname}{参考文献}
\renewcommand{\figurename}{图}
\renewcommand{\tablename}{表}
\renewcommand{\today}{\number\year 年 \number\month 月 \number\day 日}

%-------------------------封面----------------
\begin{titlepage}
  \begin{center}
    \includegraphics[width=0.8\textwidth]{NKU.png}\\[1cm]
    \vspace{20mm}
    \textbf{\huge\textbf{\kaishu{计算机学院}}}\\[0.5cm]
    \textbf{\huge{\kaishu{并行程序设计第 5 次作业}}}\\[2.3cm]
    \textbf{\Huge\textbf{\kaishu{高斯消去法的 OpenMP 并行化}}}

    \vspace{\fill}

    % \textbf{\Large \textbf{并行程序设计期末实验报告}}\\[0.8cm]
    % \HRule \\[0.9cm]
    % \HRule \\[2.0cm]
    \centering
    \textsc{\LARGE \kaishu{姓名\ :\ 丁屹}}\\[0.5cm]
    \textsc{\LARGE \kaishu{学号\ :\ 2013280}}\\[0.5cm]
    \textsc{\LARGE \kaishu{专业\ :\ 计算机科学与技术}}\\[0.5cm]

    \vfill
    {\Large \today}
  \end{center}
\end{titlepage}

\renewcommand {\thefigure}{\thesection{}.\arabic{figure}}%图片按章标号
\renewcommand{\figurename}{图}
\renewcommand{\contentsname}{目录}
\cfoot{\thepage\ of \pageref{LastPage}}%当前页 of 总页数


% 生成目录
\clearpage
\tableofcontents
\newpage

\section{问题描述}
高斯消去的计算模式如图 \ref{gauss} 所示,在第$k$步时,对第$k$行从$(k, k)$开始进行除法操作,并且将后续的$k + 1$至$N$行进行减去第$k$行的操作,串行算法如下面伪代码所示。
\begin{figure}
  \centering
  \includegraphics[width=1\textwidth]{gauss.png}
  \caption{高斯消去法示意图}
  \label{gauss}
\end{figure}
\begin{breakablealgorithm}
  \caption{普通高斯消元算法伪代码}
  \begin{algorithmic}[1] %每行显示行号  
    \Function {LU}{}
    \For {$k:=0$\ \textbf{to}\ $n$}
    \For {$j:=k+1$\ \textbf{to}\ $n$}
    \State {$A[k,j]:=A[k,j]/A[k,k]$}
    \EndFor
    \State{$A[k,k]:=1.0$}
    \For {$i:=k+1$\ \textbf{to}\ $n$}
    \For {$j:=k+1$\ \textbf{to}\ $n$}
    \State {$A[i,j]:=A[i,j]-A[i,k]*A[k,j]$}
    \EndFor
    \State{$A[i,k]:=0$}
    \EndFor
    \EndFor
    \EndFunction
  \end{algorithmic}
\end{breakablealgorithm}

观察高斯消去算法,注意到伪代码第 4,5 行第一个内嵌循环中的$A[k, j] := A[k, j]/A[k, k]$以及伪代码第$8,9,10$行双层$for$循环中的$A[i, j] := A[i, j] − A[i, k]×A[k, j]$都是可以进行向量化的循环。可以通过 OpenMP 以及 SIMD 扩展指令对这两步进行并行优化。

\section{OpenMP 算法设计}

源码链接:
\url{https://github.com/ArcanusNEO/Parallel-Programming/tree/master/5}

\subsection{测试用例的确定}

由于测试数据集较大,不便于各个平台同步,所以采用固定随机数种子为 $12345687$ 的 mt19937随机数生成器。
经过实验发现不同规模下,所有元素独立生成,限制大小在 $[0, 100]$,能够生成可以被正确消元的矩阵。

代码如下:

\begin{lstlisting}[title=测试数据集生成器,frame=trbl,language={C++}]
  uniform_real_distribution<float> dist(0, 100);
  mt19937       mt(12345687);
  int           n;
  istringstream iss(argv[1]);
  iss >> n;
  cout << n << endl;
  for (int i = 1; i <= n; ++i)
    for (int j = 1; j <= n; ++j) cout << dist(mt) << " \n"[j == n];
\end{lstlisting}

\subsection{实验环境和相关配置}

实验在华为鲲鹏 ARM 集群平台和本地 Arch Linux x86\_64 平台完成;

华为鲲鹏 ARM 集群平台使用华为毕昇 clang++ 编译器,本地 Arch Linux x86\_64 平台使用 GNU GCC 编译器;

使用 cmake 构建项目,编译开关如下:

\begin{minted}[mathescape,
  linenos,
  numbersep=5pt,
  gobble=2,
  frame=lines,
  framesep=2mm,
  highlightcolor=green!40]{cmake}
  set(CMAKE_CXX_FLAGS_RELEASE "-O3")
  set(THREADS_PREFER_PTHREAD_FLAG ON)
  find_package(OpenMP REQUIRED)
\end{minted}

实验测试了 4、8、16 线程并行的运行数据。

\subsection{算法设计}

\subsubsection{默认平凡算法}

使用一维数组模拟矩阵,避免改变矩阵大小时第二维不方便调整、必须设成最大值的问题,可以减少 cache 失效;

使用 $\#define\ matrix(i, j)\ arr[(i) * n + (j)]$ 宏,增强可读性;

\begin{lstlisting}[title=平凡算法,frame=trbl,language={C++}]
#define matrix(i, j) arr[(i) * n + (j)]
void func(int& ans, float arr[], int n) {
  for (int k = 0; k < n; ++k) {
    for (int j = k + 1; j < n; ++j) matrix(k, j) = matrix(k, j) / matrix(k, k);
    matrix(k, k) = 1.0;
    for (int i = k + 1; i < n; ++i) {
      for (int j = k + 1; j < n; ++j)
        matrix(i, j) = matrix(i, j) - matrix(i, k) * matrix(k, j);
      matrix(i, k) = 0;
    }
  }
#undef matrix
}
\end{lstlisting}

% Please add the following required packages to your document preamble:
% \usepackage{graphicx}
\begin{table}[]
  \centering
  \resizebox{\textwidth}{!}{%
    \begin{tabular}{|llll|}
      \hline
      Scale       & Reperat times & x86 ordinary (s) & arm ordinary (s) \\ \hline
      8 × 8       & 100           & 0.000001330460   & 0.000000525400   \\ \hline
      16 × 16     & 50            & 0.000001706920   & 0.000001666000   \\ \hline
      32 × 32     & 50            & 0.000003640080   & 0.000007127000   \\ \hline
      64 × 64     & 20            & 0.000015253300   & 0.000037566500   \\ \hline
      128 × 128   & 15            & 0.000098880800   & 0.000231574000   \\ \hline
      256 × 256   & 10            & 0.000716408500   & 0.001820356000   \\ \hline
      512 × 512   & 10            & 0.006722607300   & 0.014974396000   \\ \hline
      1024 × 1024 & 5             & 0.064893815400   & 0.135511226000   \\ \hline
      2048 × 2048 & 3             & 1.400074583333   & 1.101775523333   \\ \hline
      4096 × 4096 & 1             & 10.705585484000  & 13.088073440000  \\ \hline
    \end{tabular}%
  }
  \caption{所有平台平凡算法结果对比}
  \label{tab:ord}
\end{table}

\subsubsection{所有平台下只使用 OpenMP 4、8、16 线程并行化加速}

\begin{lstlisting}[title=OpenMP 并行化加速,frame=trbl,language={C++}]
  #define NUM_THREADS 8

  void func(int& ans, float arr[], int n) {
  #define matrix(i, j) arr[(i) *n + (j)]
    int   i, j, k;
    float tmp;
  #pragma omp parallel num_threads(NUM_THREADS), private(i, j, k, tmp)
    for (k = 0; k < n; ++k) {
      tmp = matrix(k, k);
  #pragma omp for
      for (j = k + 1; j < n; ++j) matrix(k, j) = matrix(k, j) / tmp;
      matrix(k, k) = 1.0;
  #pragma omp for
      for (i = k + 1; i < n; ++i) {
        tmp = matrix(i, k);
        for (j = k + 1; j < n; ++j)
          matrix(i, j) = matrix(i, j) - tmp * matrix(k, j);
        matrix(i, k) = 0;
      }
    }
  #undef matrix
  }
\end{lstlisting}

其中修改 NUM\_THREADS 宏定义可以指定不同的并行线程数。

测试了 4、8、16 的数据。

% Please add the following required packages to your document preamble:
% \usepackage{graphicx}
\begin{table}[]
  \centering
  \resizebox{\textwidth}{!}{%
    \begin{tabular}{|llllllll|}
      \hline
      Scale       & Reperat times & x86 OpenMP 4 threads (s) & x86 OpenMP 8 threads (s) & x86 OpenMP 16 threads (s) & arm OpenMP 4 threads (s) & arm OpenMP 8 threads (s) & arm OpenMP 16 threads (s) \\ \hline
      8 × 8       & 100           & 0.000007841100           & 0.000011223470           & 0.000016886970            & 0.000092409600           & 0.000126543900           & 0.000168758200            \\ \hline
      16 × 16     & 50            & 0.000016620760           & 0.000019207720           & 0.000048819040            & 0.000179391200           & 0.000235206200           & 0.000287643600            \\ \hline
      32 × 32     & 50            & 0.000029769140           & 0.000038458880           & 0.000097313150            & 0.000357201000           & 0.000496609800           & 0.000712039600            \\ \hline
      64 × 64     & 20            & 0.000063419300           & 0.000077604000           & 0.000145570260            & 0.000716369500           & 0.000993004500           & 0.001336811000            \\ \hline
      128 × 128   & 15            & 0.000164951200           & 0.000178723933           & 0.000232990467            & 0.001481558000           & 0.001898954667           & 0.002348973333            \\ \hline
      256 × 256   & 10            & 0.000460212000           & 0.000523160100           & 0.001742868100            & 0.003256248000           & 0.004177293000           & 0.005598914000            \\ \hline
      512 × 512   & 10            & 0.001627336600           & 0.001560966700           & 0.051130604200            & 0.009262748000           & 0.009874281000           & 0.011944158000            \\ \hline
      1024 × 1024 & 5             & 0.012059842800           & 0.012384870200           & 0.743364184400            & 0.045375618000           & 0.031736468000           & 0.028419450000            \\ \hline
      2048 × 2048 & 3             & 1.195753347667           & 0.823777882667           & 2.923949854667            & 0.334706430000           & 0.275254946667           & 0.112933620000            \\ \hline
      4096 × 4096 & 1             & 11.418539925000          & 9.865224839000           & 11.725731761000           & 6.097139010000           & 3.971935890000           & 2.177672700000            \\ \hline
    \end{tabular}%
  }
  \caption{所有平台 OpenMP only 结果对比}
  \label{tab:openmp}
\end{table}

\subsubsection{使用 OpenMP 及 SIMD 在 x86 平台 4、8 线程并行化加速}

\begin{lstlisting}[title=x86 OpenMP + AVX 并行化加速,frame=trbl,language={C++}]
  #define NUM_THREADS 8

  void func(int& ans, float arr[], int n) {
  #define matrix(i, j)  arr[(i) *n + (j)]
  #define pmatrix(i, j) (arr + (i * n + j))
    int    i, j, k;
    float  tmp;
    __m256 vaik, vakj, vaij, vx;
  #pragma omp parallel num_threads(NUM_THREADS), \
    private(i, j, k, tmp, vaik, vakj, vaij, vx)
    for (k = 0; k < n; ++k) {
      tmp = matrix(k, k);
  #pragma omp for
      for (j = k + 1; j < n; ++j) matrix(k, j) = matrix(k, j) / tmp;
      matrix(k, k) = 1.0;
  #pragma omp for
      for (i = k + 1; i < n; ++i) {
        vaik =
          _mm256_set_ps(matrix(i, k), matrix(i, k), matrix(i, k), matrix(i, k),
                        matrix(i, k), matrix(i, k), matrix(i, k), matrix(i, k));
        for (j = k + 1; j + 8 <= n; j += 8) {
          vakj = _mm256_loadu_ps(pmatrix(k, j));
          vaij = _mm256_loadu_ps(pmatrix(i, j));
          vx   = _mm256_mul_ps(vakj, vaik);
          vaij = _mm256_sub_ps(vaij, vx);
          _mm256_storeu_ps(pmatrix(i, j), vaij);
        }
        tmp = matrix(i, k);
        for (; j < n; ++j) matrix(i, j) = matrix(i, j) - tmp * matrix(k, j);
        matrix(i, k) = 0;
      }
    }
  #undef matrix
  }
\end{lstlisting}

其中修改 NUM\_THREADS 宏定义可以指定不同的并行线程数。

测试了 4、8 的数据。

% Please add the following required packages to your document preamble:
% \usepackage{graphicx}
\begin{table}[]
  \centering
  \resizebox{\textwidth}{!}{%
    \begin{tabular}{|llll|}
      \hline
      Scale       & Reperat times & x86 OpenMP 4 threads AVX (s) & x86 OpenMP 8 threads AVX (s) \\ \hline
      8 × 8       & 100           & 0.000009138760               & 0.000011101290               \\ \hline
      16 × 16     & 50            & 0.000015564840               & 0.000018825000               \\ \hline
      32 × 32     & 50            & 0.000030945160               & 0.000039081760               \\ \hline
      64 × 64     & 20            & 0.000067078900               & 0.000079877600               \\ \hline
      128 × 128   & 15            & 0.000156863733               & 0.000172293600               \\ \hline
      256 × 256   & 10            & 0.000542010200               & 0.000501830600               \\ \hline
      512 × 512   & 10            & 0.001857792100               & 0.001428114600               \\ \hline
      1024 × 1024 & 5             & 0.015659546200               & 0.011327669000               \\ \hline
      2048 × 2048 & 3             & 1.033535896333               & 0.737357611000               \\ \hline
      4096 × 4096 & 1             & 10.552043834000              & 10.662585467000              \\ \hline
    \end{tabular}%
  }
  \caption{x86 平台 OpenMP + AVX 结果对比}
  \label{tab:openmpavx}
\end{table}

\subsubsection{使用 OpenMP 及 SIMD 在 arm 平台 4、8 线程并行化加速}

\begin{lstlisting}[title=arm OpenMP + NEON 并行化加速,frame=trbl,language={C++}]
  #define NUM_THREADS 8

  void func(int& ans, float arr[], int n) {
  #define matrix(i, j)  arr[(i) *n + (j)]
  #define pmatrix(i, j) (arr + (i * n + j))
    int         i, j, k;
    float       tmp;
    float32x4_t vaik, vakj, vaij, vx;
  #pragma omp parallel num_threads(NUM_THREADS), \
    private(i, j, k, tmp, vaik, vakj, vaij, vx)
    for (k = 0; k < n; ++k) {
      tmp = matrix(k, k);
  #pragma omp for
      for (j = k + 1; j < n; ++j) matrix(k, j) = matrix(k, j) / tmp;
      matrix(k, k) = 1.0;
  #pragma omp for
      for (i = k + 1; i < n; ++i) {
        vaik = vdupq_n_f32(matrix(i, k));
        for (j = k + 1; j + 4 <= n; j += 4) {
          vakj = vld1q_f32(pmatrix(k, j));
          vaij = vld1q_f32(pmatrix(i, j));
          vx   = vmulq_f32(vakj, vaik);
          vaij = vsubq_f32(vaij, vx);
          vst1q_f32(pmatrix(i, j), vaij);
        }
        tmp = matrix(i, k);
        for (; j < n; ++j) matrix(i, j) = matrix(i, j) - tmp * matrix(k, j);
        matrix(i, k) = 0;
      }
    }
  #undef matrix
}
\end{lstlisting}

其中修改 NUM\_THREADS 宏定义可以指定不同的并行线程数。

测试了 4、8 的数据。

% Please add the following required packages to your document preamble:
% \usepackage{graphicx}
\begin{table}[]
  \centering
  \resizebox{\textwidth}{!}{%
    \begin{tabular}{|llll|}
      \hline
      Scale       & Reperat times & arm OpenMP 4 threads NEON (s) & arm OpenMP 8 threads NEON (s) \\ \hline
      8 × 8       & 100           & 0.000096044400                & 0.000150384800                \\ \hline
      16 × 16     & 50            & 0.000188844800                & 0.000279533400                \\ \hline
      32 × 32     & 50            & 0.000385884200                & 0.000604768200                \\ \hline
      64 × 64     & 20            & 0.000774845500                & 0.001232058000                \\ \hline
      128 × 128   & 15            & 0.001617558000                & 0.002336769333                \\ \hline
      256 × 256   & 10            & 0.003750217000                & 0.005148563000                \\ \hline
      512 × 512   & 10            & 0.011932396000                & 0.012737347000                \\ \hline
      1024 × 1024 & 5             & 0.062839944000                & 0.043704944000                \\ \hline
      2048 × 2048 & 3             & 0.504436153333                & 0.303472556667                \\ \hline
      4096 × 4096 & 1             & 7.025737910000                & 4.243570140000                \\ \hline
    \end{tabular}%
  }
  \caption{arm 平台 OpenMP + NEON 结果对比}
  \label{tab:openmpneon}
\end{table}

\section{实验结果分析}

% Please add the following required packages to your document preamble:
% \usepackage{graphicx}
\begin{table}[]
  \centering
  \resizebox{\textwidth}{!}{%
    \begin{tabular}{|llllllll|}
      \hline
      Scale       & Reperat times & x86 ordinary (s) & x86 OpenMP 4 threads (s) & x86 OpenMP 8 threads (s) & x86 OpenMP 16 threads (s) & x86 OpenMP 4 threads AVX (s) & x86 OpenMP 8 threads AVX (s) \\ \hline
      8 × 8       & 100           & 0.00000133046    & 0.0000078411             & 0.00001122347            & 0.00001688697             & 0.00000913876                & 0.00001110129                \\ \hline
      16 × 16     & 50            & 0.00000170692    & 0.00001662076            & 0.00001920772            & 0.00004881904             & 0.00001556484                & 0.000018825                  \\ \hline
      32 × 32     & 50            & 0.00000364008    & 0.00002976914            & 0.00003845888            & 0.00009731315             & 0.00003094516                & 0.00003908176                \\ \hline
      64 × 64     & 20            & 0.0000152533     & 0.0000634193             & 0.000077604              & 0.00014557026             & 0.0000670789                 & 0.0000798776                 \\ \hline
      128 × 128   & 15            & 0.0000988808     & 0.0001649512             & 0.000178723933           & 0.000232990467            & 0.000156863733               & 0.0001722936                 \\ \hline
      256 × 256   & 10            & 0.0007164085     & 0.000460212              & 0.0005231601             & 0.0017428681              & 0.0005420102                 & 0.0005018306                 \\ \hline
      512 × 512   & 10            & 0.0067226073     & 0.0016273366             & 0.0015609667             & 0.0511306042              & 0.0018577921                 & 0.0014281146                 \\ \hline
      1024 × 1024 & 5             & 0.0648938154     & 0.0120598428             & 0.0123848702             & 0.7433641844              & 0.0156595462                 & 0.011327669                  \\ \hline
      2048 × 2048 & 3             & 1.400074583333   & 1.195753347667           & 0.823777882667           & 2.923949854667            & 1.033535896333               & 0.737357611                  \\ \hline
      4096 × 4096 & 1             & 10.705585484     & 11.418539925             & 9.865224839              & 11.725731761              & 10.552043834                 & 10.662585467                 \\ \hline
    \end{tabular}%
  }
  \caption{x86 平台所有结果对比}
  \label{tab:x86}
\end{table}

% Please add the following required packages to your document preamble:
% \usepackage{graphicx}
\begin{table}[]
  \centering
  \resizebox{\textwidth}{!}{%
    \begin{tabular}{|llllllll|}
      \hline
      Scale       & Reperat times & arm ordinary (s) & arm OpenMP 4 threads (s) & arm OpenMP 8 threads (s) & arm OpenMP 16 threads (s) & arm OpenMP 4 threads NEON (s) & arm OpenMP 8 threads NEON (s) \\ \hline
      8 × 8       & 100           & 0.0000005254     & 0.0000924096             & 0.0001265439             & 0.0001687582              & 0.0000960444                  & 0.0001503848                  \\ \hline
      16 × 16     & 50            & 0.000001666      & 0.0001793912             & 0.0002352062             & 0.0002876436              & 0.0001888448                  & 0.0002795334                  \\ \hline
      32 × 32     & 50            & 0.000007127      & 0.000357201              & 0.0004966098             & 0.0007120396              & 0.0003858842                  & 0.0006047682                  \\ \hline
      64 × 64     & 20            & 0.0000375665     & 0.0007163695             & 0.0009930045             & 0.001336811               & 0.0007748455                  & 0.001232058                   \\ \hline
      128 × 128   & 15            & 0.000231574      & 0.001481558              & 0.001898954667           & 0.002348973333            & 0.001617558                   & 0.002336769333                \\ \hline
      256 × 256   & 10            & 0.001820356      & 0.003256248              & 0.004177293              & 0.005598914               & 0.003750217                   & 0.005148563                   \\ \hline
      512 × 512   & 10            & 0.014974396      & 0.009262748              & 0.009874281              & 0.011944158               & 0.011932396                   & 0.012737347                   \\ \hline
      1024 × 1024 & 5             & 0.135511226      & 0.045375618              & 0.031736468              & 0.02841945                & 0.062839944                   & 0.043704944                   \\ \hline
      2048 × 2048 & 3             & 1.101775523333   & 0.33470643               & 0.275254946667           & 0.11293362                & 0.504436153333                & 0.303472556667                \\ \hline
      4096 × 4096 & 1             & 13.08807344      & 6.09713901               & 3.97193589               & 2.1776727                 & 7.02573791                    & 4.24357014                    \\ \hline
    \end{tabular}%
  }
  \caption{arm 平台所有结果对比}
  \label{tab:arm}
\end{table}

\begin{figure}[h]
  \centering
  \includegraphics[width=\textwidth]{all.pdf}
  \caption{所有平台所有结果对比折线图}
  \label{pic:all}
\end{figure}

\begin{figure}[h]
  \centering
  \includegraphics[width=\textwidth]{x86.pdf}
  \caption{x86 平台所有结果对比折线图}
  \label{pic:x86}
\end{figure}

\begin{figure}[h]
  \centering
  \includegraphics[width=\textwidth]{arm.pdf}
  \caption{arm 平台所有结果对比折线图}
  \label{pic:arm}
\end{figure}

对比表格 \ref{tab:ord}、\ref{tab:openmp}、\ref{tab:openmpavx}、\ref{tab:openmpneon}、\ref{tab:x86}、\ref{tab:arm} 可以发现,
对于 arm 平台,附加的 SIMD 加速效果不好,但是 OpenMP 加速效果很好:最快的 4 线程对比平凡算法最大数据加速比达到了 13.08807344 ÷ 6.09713901 = 2.146592587,
最快的 8 线程对比平凡算法最大数据加速比达到了 13.08807344 ÷ 3.97193589 = 3.295137123,
最快的 16 线程对比平凡算法最大数据加速比达到了 13.08807344 ÷ 2.1776727 = 6.010119611。

对于 x86 平台,加速效果与数据规模关系很大,在小数据量和大数据量下平凡算法与优化算法速度差异不大,在中等数据量下加速比有时能超越核心数的增加;
值得注意的是, x86 平台的 16 线程加速效果很差,原因可能与处理器架构有关:使用了 AMD Ryzen 4800HS 的 CPU,8C16T,使用 1 个 CCD 2 个 CCX,两个 CCX 间不共享三级缓存,导致 16 线程加速带来过多的缓存失效。

% \newpage

% \section{参考文献}
% \cite{1}\cite{2}\cite{3}\cite{4}\cite{5}\cite{6}

% \bibliographystyle{plain}
% \bibliography{Parallel-Programming-0.bib}

\end{document}