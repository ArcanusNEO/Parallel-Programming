\documentclass[a4paper]{article}

\input{style/ch_xelatex.tex}
\input{style/scala.tex}

%代码段设置
\lstset{numbers=left,
basicstyle=\tiny,
numberstyle=\tiny,
keywordstyle=\color{blue!70},
commentstyle=\color{red!50!green!50!blue!50},
frame=single, rulesepcolor=\color{red!20!green!20!blue!20},
escapeinside=``
}

\graphicspath{ {figures/} }
\usepackage{ctex}
\usepackage{graphicx}
\usepackage{color,framed}%文本框
\usepackage{listings}
\usepackage{caption}
\usepackage{amssymb}
\usepackage{enumerate}
\usepackage{xcolor}
\usepackage{bm} 
\usepackage{lastpage}%获得总页数
\usepackage{fancyhdr}
\usepackage{tabularx}  
\usepackage{geometry}
\usepackage{minted}
\usepackage{graphics}
\usepackage{subfigure}
\usepackage{float}
\usepackage{pdfpages}
\usepackage{pgfplots}
\pgfplotsset{width=10cm,compat=1.9}
\usepackage{multirow}
\usepackage{footnote}
\usepackage{booktabs}
\usepackage{url}
\usepackage{underscore}

%-----------------------伪代码------------------
\usepackage{algorithm}  
\usepackage{algorithmicx}  
\usepackage{algpseudocode}  
\floatname{algorithm}{Algorithm}  
\renewcommand{\algorithmicrequire}{\textbf{Input:}}  
\renewcommand{\algorithmicensure}{\textbf{Output:}} 
\usepackage{lipsum}  
\makeatletter
\newenvironment{breakablealgorithm}
  {% \begin{breakablealgorithm}
  \begin{center}
     \refstepcounter{algorithm}% New algorithm
     \hrule height.8pt depth0pt \kern2pt% \@fs@pre for \@fs@ruled
     \renewcommand{\caption}[2][\relax]{% Make a new \caption
      {\raggedright\textbf{\ALG@name~\thealgorithm} ##2\par}%
      \ifx\relax##1\relax % #1 is \relax
         \addcontentsline{loa}{algorithm}{\protect\numberline{\thealgorithm}##2}%
      \else % #1 is not \relax
         \addcontentsline{loa}{algorithm}{\protect\numberline{\thealgorithm}##1}%
      \fi
      \kern2pt\hrule\kern2pt
     }
  }{% \end{breakablealgorithm}
     \kern2pt\hrule\relax% \@fs@post for \@fs@ruled
  \end{center}
  }
\makeatother
%------------------------代码-------------------
\usepackage{xcolor} 
\usepackage{listings} 
\lstset{ 
breaklines,%自动换行
basicstyle=\small,
escapeinside=``,
keywordstyle=\color{ blue!70} \bfseries,
commentstyle=\color{red!50!green!50!blue!50},% 
stringstyle=\ttfamily,% 
extendedchars=false,% 
linewidth=\textwidth,% 
numbers=left,% 
numberstyle=\tiny \color{blue!50},% 
frame=trbl% 
rulesepcolor= \color{ red!20!green!20!blue!20} 
}

%-------------------------页面边距--------------
\geometry{a4paper,left=2.3cm,right=2.3cm,top=2.7cm,bottom=2.7cm}
%-------------------------页眉页脚--------------
\usepackage{fancyhdr}
\pagestyle{fancy}
\lhead{\kaishu \leftmark}
% \chead{}
\rhead{\kaishu 并行程序设计实验报告}%加粗\bfseries 
\lfoot{}
\cfoot{\thepage}
\rfoot{}
\renewcommand{\headrulewidth}{0.1pt}  
\renewcommand{\footrulewidth}{0pt}%去掉横线
\newcommand{\HRule}{\rule{\linewidth}{0.5mm}}%标题横线
\newcommand{\HRulegrossa}{\rule{\linewidth}{1.2mm}}
\setlength{\textfloatsep}{10mm}%设置图片的前后间距
%--------------------文档内容--------------------

\begin{document}
\renewcommand{\contentsname}{目\ 录}
\renewcommand{\appendixname}{附录}
\renewcommand{\appendixpagename}{附录}
\renewcommand{\refname}{参考文献}
\renewcommand{\figurename}{图}
\renewcommand{\tablename}{表}
\renewcommand{\today}{\number\year 年 \number\month 月 \number\day 日}

%-------------------------封面----------------
\begin{titlepage}
  \begin{center}
    \includegraphics[width=0.8\textwidth]{NKU.png}\\[1cm]
    \vspace{20mm}
    \textbf{\huge\textbf{\kaishu{计算机学院}}}\\[0.5cm]
    \textbf{\huge{\kaishu{并行程序设计第2.1次作业}}}\\[2.3cm]
    \textbf{\Huge\textbf{\kaishu{矩阵与向量内积}}}

    \vspace{\fill}

    % \textbf{\Large \textbf{并行程序设计期末实验报告}}\\[0.8cm]
    % \HRule \\[0.9cm]
    % \HRule \\[2.0cm]
    \centering
    \textsc{\LARGE \kaishu{姓名\ :\ 丁屹}}\\[0.5cm]
    \textsc{\LARGE \kaishu{学号\ :\ 2013280}}\\[0.5cm]
    \textsc{\LARGE \kaishu{专业\ :\ 计算机科学与技术}}\\[0.5cm]

    \vfill
    {\Large \today}
  \end{center}
\end{titlepage}

\renewcommand {\thefigure}{\thesection{}.\arabic{figure}}%图片按章标号
\renewcommand{\figurename}{图}
\renewcommand{\contentsname}{目录}
\cfoot{\thepage\ of \pageref{LastPage}}%当前页 of 总页数


% 生成目录
\clearpage
\tableofcontents
\newpage

\section{问题}
计算给定 n × n 矩阵的每一列与给定向量的内积,考虑两种算法设计思路:
\begin{enumerate}
  \item 逐列访问元素的平凡算法
  \item cache 优化算法
\end{enumerate}

\section{程序实现}
源码链接:
\url{https://github.com/ArcanusNEO/Parallel-Programming/tree/master/1/0}

头文件位于 inc/,源文件位于 src/

\begin{lstlisting}[title=逐列访问平凡算法,frame=trbl,language={C++}]
  for (int i = 1; i <= matrix.col(); ++i)
    for (int j = 1; j <= matrix.row(); ++j)
      ans += (long long) matrix(j, i) * vec(i);
\end{lstlisting}

\begin{lstlisting}[title=cache 优化算法,frame=trbl,language={C++}]
  for (int i = 1; i <= matrix.row(); ++i)
    for (int j = 1; j <= matrix.col(); ++j)
      ans += (long long) matrix(i, j) * vec(j);
\end{lstlisting}

\begin{itemize}
  \item 为了便于调整数据规模的同时保证数据分布紧凑,矩阵采用一维数组模拟,封装到 matrix_t 类中,使用 operator() 访问元素
  \item 使用 C++11 的 chrono::high_resolution_clock 高精度计时函数测量运行时间
  \item ordinary 采用逐列访问的平凡算法
  \item cache 采用 cache 优化算法
  \item 使用 cmake 构建
\end{itemize}

\section{实验平台配置}
在华为鲲鹏arm平台部分参数如表 \ref{fig:1} 所示。
\begin{table}[]
  \centering
  \begin{tabular}{l l}
    \hline
    CPU Maximum Frequency & 2600 MHz  \\   \hline
    CPU Minimum Frequency & 200 MHz   \\   \hline
    L1d 缓存              & 64 KiB    \\   \hline
    L1i 缓存              & 64 KiB    \\   \hline
    L2 缓存               & 512 KiB   \\   \hline
    L3 缓存               & 49152 KiB \\   \hline
    内存大小              & 191.3 GiB \\   \hline
  \end{tabular}
  \caption{鲲鹏服务器硬件配置信息}
  \label{fig:1}
\end{table}

\section{测试数据}
使用 gen-data 生成数据,规模 n 作为第一个参数传入,使用 mt1937 生成随机数。

生成了一组测试文件位于 res/,文件名形如 n.in

使用 conf/in.conf 配置输入数据路径和重复测试次数,其路径作为待测程序的第一个参数传入。

\section{}

\newpage

\section{参考文献}
\cite{1}\cite{2}\cite{3}\cite{4}\cite{5}\cite{6}

\bibliographystyle{plain}
\bibliography{Parallel-Programming-0.bib}

\end{document}