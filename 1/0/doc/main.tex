\documentclass[a4paper]{article}

\input{style/ch_xelatex.tex}
\input{style/scala.tex}

%代码段设置
\lstset{numbers=left,
basicstyle=\tiny,
numberstyle=\tiny,
keywordstyle=\color{blue!70},
commentstyle=\color{red!50!green!50!blue!50},
frame=single, rulesepcolor=\color{red!20!green!20!blue!20},
escapeinside=``
}

\graphicspath{ {figures/} }
\usepackage{ctex}
\usepackage{graphicx}
\usepackage{color,framed}%文本框
\usepackage{listings}
\usepackage{caption}
\usepackage{amssymb}
\usepackage{enumerate}
\usepackage{xcolor}
\usepackage{bm} 
\usepackage{lastpage}%获得总页数
\usepackage{fancyhdr}
\usepackage{tabularx}  
\usepackage{geometry}
\usepackage{minted}
\usepackage{graphics}
\usepackage{subfigure}
\usepackage{float}
\usepackage{pdfpages}
\usepackage{pgfplots}
\pgfplotsset{width=10cm,compat=1.9}
\usepackage{multirow}
\usepackage{footnote}
\usepackage{booktabs}
\usepackage{url}
\usepackage{underscore}

%-----------------------伪代码------------------
\usepackage{algorithm}  
\usepackage{algorithmicx}  
\usepackage{algpseudocode}  
\floatname{algorithm}{Algorithm}  
\renewcommand{\algorithmicrequire}{\textbf{Input:}}  
\renewcommand{\algorithmicensure}{\textbf{Output:}} 
\usepackage{lipsum}  
\makeatletter
\newenvironment{breakablealgorithm}
  {% \begin{breakablealgorithm}
  \begin{center}
     \refstepcounter{algorithm}% New algorithm
     \hrule height.8pt depth0pt \kern2pt% \@fs@pre for \@fs@ruled
     \renewcommand{\caption}[2][\relax]{% Make a new \caption
      {\raggedright\textbf{\ALG@name~\thealgorithm} ##2\par}%
      \ifx\relax##1\relax % #1 is \relax
         \addcontentsline{loa}{algorithm}{\protect\numberline{\thealgorithm}##2}%
      \else % #1 is not \relax
         \addcontentsline{loa}{algorithm}{\protect\numberline{\thealgorithm}##1}%
      \fi
      \kern2pt\hrule\kern2pt
     }
  }{% \end{breakablealgorithm}
     \kern2pt\hrule\relax% \@fs@post for \@fs@ruled
  \end{center}
  }
\makeatother
%------------------------代码-------------------
\usepackage{xcolor} 
\usepackage{listings} 
\lstset{ 
breaklines,%自动换行
basicstyle=\small,
escapeinside=``,
keywordstyle=\color{ blue!70} \bfseries,
commentstyle=\color{red!50!green!50!blue!50},% 
stringstyle=\ttfamily,% 
extendedchars=false,% 
linewidth=\textwidth,% 
numbers=left,% 
numberstyle=\tiny \color{blue!50},% 
frame=trbl% 
rulesepcolor= \color{ red!20!green!20!blue!20} 
}

%-------------------------页面边距--------------
\geometry{a4paper,left=2.3cm,right=2.3cm,top=2.7cm,bottom=2.7cm}
%-------------------------页眉页脚--------------
\usepackage{fancyhdr}
\pagestyle{fancy}
\lhead{\kaishu \leftmark}
% \chead{}
\rhead{\kaishu 并行程序设计实验报告}%加粗\bfseries 
\lfoot{}
\cfoot{\thepage}
\rfoot{}
\renewcommand{\headrulewidth}{0.1pt}  
\renewcommand{\footrulewidth}{0pt}%去掉横线
\newcommand{\HRule}{\rule{\linewidth}{0.5mm}}%标题横线
\newcommand{\HRulegrossa}{\rule{\linewidth}{1.2mm}}
\setlength{\textfloatsep}{10mm}%设置图片的前后间距
%--------------------文档内容--------------------

\begin{document}
\renewcommand{\contentsname}{目\ 录}
\renewcommand{\appendixname}{附录}
\renewcommand{\appendixpagename}{附录}
\renewcommand{\refname}{参考文献}
\renewcommand{\figurename}{图}
\renewcommand{\tablename}{表}
\renewcommand{\today}{\number\year 年 \number\month 月 \number\day 日}

%-------------------------封面----------------
\begin{titlepage}
  \begin{center}
    \includegraphics[width=0.8\textwidth]{NKU.png}\\[1cm]
    \vspace{20mm}
    \textbf{\huge\textbf{\kaishu{计算机学院}}}\\[0.5cm]
    \textbf{\huge{\kaishu{并行程序设计第2.1次作业}}}\\[2.3cm]
    \textbf{\Huge\textbf{\kaishu{矩阵与向量内积}}}

    \vspace{\fill}

    % \textbf{\Large \textbf{并行程序设计期末实验报告}}\\[0.8cm]
    % \HRule \\[0.9cm]
    % \HRule \\[2.0cm]
    \centering
    \textsc{\LARGE \kaishu{姓名\ :\ 丁屹}}\\[0.5cm]
    \textsc{\LARGE \kaishu{学号\ :\ 2013280}}\\[0.5cm]
    \textsc{\LARGE \kaishu{专业\ :\ 计算机科学与技术}}\\[0.5cm]

    \vfill
    {\Large \today}
  \end{center}
\end{titlepage}

\renewcommand {\thefigure}{\thesection{}.\arabic{figure}}%图片按章标号
\renewcommand{\figurename}{图}
\renewcommand{\contentsname}{目录}
\cfoot{\thepage\ of \pageref{LastPage}}%当前页 of 总页数


% 生成目录
\clearpage
\tableofcontents
\newpage

\section{问题}
计算给定 n × n 矩阵的每一列与给定向量的内积,考虑两种算法设计思路:
\begin{enumerate}
  \item 逐列访问元素的平凡算法
  \item cache 优化算法
\end{enumerate}

\section{程序实现}
源码链接:
\url{https://github.com/ArcanusNEO/Parallel-Programming/tree/master/1/0}

头文件位于 inc/,源文件位于 src/

\begin{lstlisting}[title=逐列访问平凡算法,frame=trbl,language={C++}]
  for (int i = 1; i <= matrix.col(); ++i)
    for (int j = 1; j <= matrix.row(); ++j)
      ans += (long long) matrix(j, i) * vec(i);
\end{lstlisting}

\begin{lstlisting}[title=cache 优化算法,frame=trbl,language={C++}]
  for (int i = 1; i <= matrix.row(); ++i)
    for (int j = 1; j <= matrix.col(); ++j)
      ans += (long long) matrix(i, j) * vec(j);
\end{lstlisting}

\begin{itemize}
  \item 为了便于调整数据规模的同时保证数据分布紧凑,矩阵采用一维数组模拟,封装到 matrix_t 类中,使用 operator() 访问元素
  \item 使用 C++11 的 chrono::high_resolution_clock 高精度计时函数测量运行时间
  \item ordinary 采用逐列访问的平凡算法
  \item cache 采用 cache 优化算法
  \item 使用 cmake 构建
\end{itemize}

\section{实验平台配置}
\begin{table}[]
  \centering
  \begin{tabular}{ll}
    \hline
    CPU Maximum Frequency & 2600 MHz  \\   \hline
    CPU Minimum Frequency & 200 MHz   \\   \hline
    L1d 缓存              & 64 KiB    \\   \hline
    L1i 缓存              & 64 KiB    \\   \hline
    L2 缓存               & 512 KiB   \\   \hline
    L3 缓存               & 49152 KiB \\   \hline
    内存大小              & 191.3 GiB \\   \hline
  \end{tabular}
  \caption{鲲鹏 arm 平台硬件配置信息}
  \label{tab:arm-arch}
\end{table}

\begin{table}[]
  \centering
  \begin{tabular}{ll}
    \hline
    型号                  & AMD Ryzen 7 4800HS with Radeon Graphics \\   \hline
    CPU Maximum Frequency & 2900 MHz                                \\   \hline
    CPU Minimum Frequency & 1400 MHz                                \\   \hline
    L1d 缓存              & 256 KiB                                 \\   \hline
    L1i 缓存              & 256 KiB                                 \\   \hline
    L2 缓存               & 4 MiB                                   \\   \hline
    L3 缓存               & 8 MiB                                   \\   \hline
    内存大小              & 16 GiB                                  \\   \hline
  \end{tabular}
  \caption{本地 x86 平台硬件配置信息}
  \label{tab:x86-arch}
\end{table}

华为鲲鹏 arm 平台部分硬件参数如表 \ref{tab:arm-arch} 所示。arm 服务器系统环境为 4.14.0 内核的 openEuler,本次实验使用基于 clang 的华为 bisheng 编译器构建。

本地 x86 平台部分硬件参数如表 \ref{tab:x86-arch} 所示。本地 x86 系统环境为 5.16.14 内核的 Arch Linux,本次实验使用 GNU GCC 编译,并且使用 perf 进行性能测试。

\section{实验方案设计}
\subsection{测试脚本}
测试脚本位于 bin/,''run'' 是 x86 架构下的脚本,''run-pbs'' 是鲲鹏服务器平台的脚本

\subsection{测试数据}
使用 gen-data 生成数据,规模 n 作为第一个参数传入,使用 mt1937 生成随机数。

生成了一组测试文件位于 res/,文件名形如 n.in

使用 conf/in.conf 配置输入数据路径和重复测试次数,其路径作为待测程序的第一个参数传入。

\subsection{测试方法}
分别测试 ordinary 和 cache 两个程序。读入定义测试输入文件路径和重复测试遍数的配置文件以便自动完成测试。
程序会向标准输出打印测试结果。

\subsection{实验结果及分析}
\begin{table}[]
  \centering
  \resizebox{\textwidth}{!}{%
    \begin{tabular}{llllllllll}
      n    & repeat & ordinary O0    & cache O0       & ordinary O1    & cache O1       & ordinary O2    & cache O2       & ordinary O3    & cache O3       \\
      10   & 1000   & 0.000003921510 & 0.000003898680 & 0.000002001930 & 0.000001964040 & 0.000000327890 & 0.000000339470 & 0.000000319410 & 0.000000344110 \\
      20   & 1000   & 0.000014620950 & 0.000014602760 & 0.000007023470 & 0.000006842360 & 0.000000501310 & 0.000000657150 & 0.000000467850 & 0.000000641370 \\
      30   & 1000   & 0.000032228360 & 0.000032184710 & 0.000015241810 & 0.000014904280 & 0.000000763690 & 0.000000774170 & 0.000000691040 & 0.000000777670 \\
      40   & 950    & 0.000056860989 & 0.000056846558 & 0.000026680400 & 0.000026098295 & 0.000001164211 & 0.000001254295 & 0.000001005126 & 0.000001245632 \\
      50   & 900    & 0.000089057611 & 0.000088343433 & 0.000041381567 & 0.000040420867 & 0.000001678011 & 0.000001547344 & 0.000001410889 & 0.000001549533 \\
      60   & 850    & 0.000128093988 & 0.000127123306 & 0.000059313106 & 0.000057921706 & 0.000002270624 & 0.000002428812 & 0.000001873753 & 0.000002413847 \\
      70   & 800    & 0.000173778637 & 0.000172767013 & 0.000080434662 & 0.000078577600 & 0.000003020475 & 0.000002645638 & 0.000002430088 & 0.000002644038 \\
      80   & 750    & 0.000227279400 & 0.000224837240 & 0.000104802547 & 0.000102348533 & 0.000004033467 & 0.000003892293 & 0.000003213960 & 0.000003860040 \\
      90   & 500    & 0.000286074980 & 0.000285214100 & 0.000132402900 & 0.000129312160 & 0.000005021300 & 0.000004114980 & 0.000003812880 & 0.000004122680 \\
      100  & 100    & 0.000353027200 & 0.000351141100 & 0.000163288700 & 0.000159662300 & 0.000005953000 & 0.000005611200 & 0.000004950000 & 0.000005456300 \\
      200  & 100    & 0.001409264900 & 0.001407091700 & 0.000649411800 & 0.000634192400 & 0.000023487600 & 0.000021635600 & 0.000020187600 & 0.000021334400 \\
      300  & 100    & 0.003158901300 & 0.003158875000 & 0.001459003600 & 0.001424701100 & 0.000049816900 & 0.000046524400 & 0.000046958800 & 0.000045225600 \\
      400  & 75     & 0.005616461067 & 0.005714338133 & 0.002596611067 & 0.002590574133 & 0.000097343200 & 0.000084646933 & 0.000085758667 & 0.000082299600 \\
      500  & 50     & 0.008824118800 & 0.008933916800 & 0.004065168800 & 0.004048300400 & 0.000160591000 & 0.000133844200 & 0.000149521400 & 0.000131648000 \\
      750  & 25     & 0.020230814800 & 0.020136468000 & 0.009124470400 & 0.009196836800 & 0.000597472000 & 0.000290120400 & 0.000539312400 & 0.000290214000 \\
      1000 & 10     & 0.035702858000 & 0.036474819000 & 0.016245198000 & 0.016802713000 & 0.000869622000 & 0.000562403000 & 0.000798831000 & 0.000522889000 \\
      2000 & 10     & 0.145383594000 & 0.150944694000 & 0.065200668000 & 0.071015338000 & 0.004459454000 & 0.002142993000 & 0.004385257000 & 0.002140263000 \\
      3000 & 5      & 0.336490798000 & 0.349886098000 & 0.156467236000 & 0.168448982000 & 0.012147040000 & 0.005809428000 & 0.011584076000 & 0.006152252000 \\
      2500 & 5      & 0.228733876000 & 0.240887894000 & 0.102909028000 & 0.113171246000 & 0.007623726000 & 0.003432526000 & 0.007216164000 & 0.003632204000 \\
      3500 & 5      & 0.458229866000 & 0.478521286000 & 0.208403398000 & 0.230962040000 & 0.015620854000 & 0.007486212000 & 0.014909178000 & 0.007414828000 \\
      4000 & 5      & 0.589321022000 & 0.625729100000 & 0.264033642000 & 0.301310122000 & 0.023590688000 & 0.009625902000 & 0.023382362000 & 0.009711832000
    \end{tabular}%
  }
  \caption{鲲鹏 arm 平台不同优化等级下的测试结果(时间单位:s)}
  \label{tab:arm-test}
\end{table}

\subsection{鲲鹏 arm 平台}
通过横向比较优化级别可以得知,对于平凡算法,优化级别从 O0 到 O1 产生了 1 倍多的性能提升,从 O1 到 O2 能产生大约了 10 倍多的性能提升,而 O3 相对于 O2 的性能提升幅度很小,效果不明显。
对于 cache 优化算法的结论也类似。

通过横向比较两种算法可以得知,对于 O0 优化,两种算法性能差别不大,甚至 cache 优化算法在大数据量下处于劣势。

在小数据量下,两种算法性能差别同样不大。

在中等数据量下,各种优化等级 cache 优化算法相比于平凡算法都有些微的性能优势。

在大数据量下,对于 O0 和 O1 优化,cache 优化算法相比于平凡算法处于劣势;
对于 O2 和 O3 优化,cache 优化算法相比于平凡算法有1倍有余的性能提升。

通过纵向比较可以得知,由于算法使用 int 存储矩阵元素,
4 × 100 × 100 B < 64 KiB = L1d size < 4 × 200 × 200 B,
所以在 n = 200 时矩阵大小突破了 L1d 缓存大小,相对于 n = 100 数据规模翻 4 倍,运行时间成 4 倍有余;
而 n = 100 与 n = 50 相比,运行时间恰好大约成 4 倍。

4 × 300 × 300 B < 512 KiB = L2 size < 4 × 400 × 400 B,
所以在 n = 400 时矩阵大小突破了 L2 缓存大小,此时低优化级别下 cache 优化算法和平凡算法性能差别不大,而高优化级别下 cache 优化算法对比于平凡算法有较大性能优势。

\newpage

\section{参考文献}
\cite{1}\cite{2}\cite{3}\cite{4}\cite{5}\cite{6}

\bibliographystyle{plain}
\bibliography{Parallel-Programming-0.bib}

\end{document}