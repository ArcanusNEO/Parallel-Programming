\documentclass[a4paper]{article}

\input{style/ch_xelatex.tex}
\input{style/scala.tex}

%代码段设置
\lstset{numbers=left,
basicstyle=\tiny,
numberstyle=\tiny,
keywordstyle=\color{blue!70},
commentstyle=\color{red!50!green!50!blue!50},
frame=single, rulesepcolor=\color{red!20!green!20!blue!20},
escapeinside=``
}

\graphicspath{ {figures/} }
\usepackage{ctex}
\usepackage{graphicx}
\usepackage{color,framed}%文本框
\usepackage{listings}
\usepackage{caption}
\usepackage{amssymb}
\usepackage{enumerate}
\usepackage{xcolor}
\usepackage{bm} 
\usepackage{lastpage}%获得总页数
\usepackage{fancyhdr}
\usepackage{tabularx}  
\usepackage{geometry}
\usepackage{minted}
\usepackage{graphics}
\usepackage{subfigure}
\usepackage{float}
\usepackage{pdfpages}
\usepackage{pgfplots}
\pgfplotsset{width=10cm,compat=1.9}
\usepackage{multirow}
\usepackage{footnote}
\usepackage{booktabs}

%-----------------------伪代码------------------
\usepackage{algorithm}  
\usepackage{algorithmicx}  
\usepackage{algpseudocode}  
\floatname{algorithm}{Algorithm}  
\renewcommand{\algorithmicrequire}{\textbf{Input:}}  
\renewcommand{\algorithmicensure}{\textbf{Output:}} 
\usepackage{lipsum}  
\makeatletter
\newenvironment{breakablealgorithm}
  {% \begin{breakablealgorithm}
  \begin{center}
     \refstepcounter{algorithm}% New algorithm
     \hrule height.8pt depth0pt \kern2pt% \@fs@pre for \@fs@ruled
     \renewcommand{\caption}[2][\relax]{% Make a new \caption
      {\raggedright\textbf{\ALG@name~\thealgorithm} ##2\par}%
      \ifx\relax##1\relax % #1 is \relax
         \addcontentsline{loa}{algorithm}{\protect\numberline{\thealgorithm}##2}%
      \else % #1 is not \relax
         \addcontentsline{loa}{algorithm}{\protect\numberline{\thealgorithm}##1}%
      \fi
      \kern2pt\hrule\kern2pt
     }
  }{% \end{breakablealgorithm}
     \kern2pt\hrule\relax% \@fs@post for \@fs@ruled
  \end{center}
  }
\makeatother
%------------------------代码-------------------
\usepackage{xcolor} 
\usepackage{listings} 
\lstset{ 
breaklines,%自动换行
basicstyle=\small,
escapeinside=``,
keywordstyle=\color{ blue!70} \bfseries,
commentstyle=\color{red!50!green!50!blue!50},% 
stringstyle=\ttfamily,% 
extendedchars=false,% 
linewidth=\textwidth,% 
numbers=left,% 
numberstyle=\tiny \color{blue!50},% 
frame=trbl% 
rulesepcolor= \color{ red!20!green!20!blue!20} 
}

%-------------------------页面边距--------------
\geometry{a4paper,left=2.3cm,right=2.3cm,top=2.7cm,bottom=2.7cm}
%-------------------------页眉页脚--------------
\usepackage{fancyhdr}
\pagestyle{fancy}
\lhead{\kaishu \leftmark}
% \chead{}
\rhead{\kaishu 并行程序设计实验报告}%加粗\bfseries 
\lfoot{}
\cfoot{\thepage}
\rfoot{}
\renewcommand{\headrulewidth}{0.1pt}  
\renewcommand{\footrulewidth}{0pt}%去掉横线
\newcommand{\HRule}{\rule{\linewidth}{0.5mm}}%标题横线
\newcommand{\HRulegrossa}{\rule{\linewidth}{1.2mm}}
\setlength{\textfloatsep}{10mm}%设置图片的前后间距
%--------------------文档内容--------------------

\begin{document}
\renewcommand{\contentsname}{目\ 录}
\renewcommand{\appendixname}{附录}
\renewcommand{\appendixpagename}{附录}
\renewcommand{\refname}{参考文献}
\renewcommand{\figurename}{图}
\renewcommand{\tablename}{表}
\renewcommand{\today}{\number\year 年 \number\month 月 \number\day 日}

%-------------------------封面----------------
\begin{titlepage}
  \begin{center}
    \includegraphics[width=0.8\textwidth]{NKU.png}\\[1cm]
    \vspace{20mm}
    \textbf{\huge\textbf{\kaishu{计算机学院}}}\\[0.5cm]
    \textbf{\huge{\kaishu{并行程序设计}}}\\[2.3cm]
    \textbf{\Huge\textbf{\kaishu{Gröbner 基计算中的高斯消元并行化改进}}}

    \vspace{\fill}

    % \textbf{\Large \textbf{并行程序设计期末实验报告}}\\[0.8cm]
    % \HRule \\[0.9cm]
    % \HRule \\[2.0cm]
    \centering
    \textsc{\LARGE \kaishu{小组成员姓名\ :\ 丁屹、卢麒萱}}\\[0.5cm]
    \textsc{\LARGE \kaishu{小组成员学号\ :\ 2013280、2010519}}\\[0.5cm]
    \textsc{\LARGE \kaishu{专业\ :\ 计算机科学与技术}}\\[0.5cm]

    \vfill
    {\Large \today}
  \end{center}
\end{titlepage}

\renewcommand {\thefigure}{\thesection{}.\arabic{figure}}%图片按章标号
\renewcommand{\figurename}{图}
\renewcommand{\contentsname}{目录}
\cfoot{\thepage\ of \pageref{LastPage}}%当前页 of 总页数

% 生成目录
\clearpage
\tableofcontents
\newpage

\section{研究问题}

\section{背景知识}

\subsection{普通高斯消去算法}

\includegraphics[width=\textwidth]{gauss-alg.jpg}

\begin{lstlisting}[title=普通高斯消去算法,frame=trbl,language={Pascal}]
  procedure LU (A)
  begin
    for k := 1 to n do
      for j := k + 1 to n do
        A[k, j] := A[k, j] / A[k, k];
      endfor;
      A[k, k] := 1.0;
      for i := k + 1 to n do
        for j := k + 1 to n do
          A[i, j] := A[i, j] - A[i, k] * A[k, j];
        endfor;
        A[i, k] := 0;
      endfor;
    endfor;
  end LU
\end{lstlisting}

观察高斯消去算法,注意到伪代码第 4,5 行第一个内嵌循环中的
A[k, j] := A[k, j] / A[k, k] 以及伪代码第 8、9、10 行双层 for 循环中的
A[i, j] := A[i, j] - A[i, k] * A[k, j] 都是可以进行向量化的循环。可以通过
SIMD 扩展指令对这两步进行并行优化。

\subsection{数据结构}

可采用位向量方式存储每个消元子和被消元行,优点是消元操作变为位向量异或操作,算法实现简单,且适合并行化,易达到更高的并行效率,缺点是 Gröbner 基计算中产生的消元子和被消元行非常稀疏,非零元素( 1 元素)在 5 \% 以下,位向量存储和计算可能并非最优。
也可采用类似倒排链表的存储及方式——可认为是稀疏 0 / 1 矩阵的紧凑存储方式,每个消元子和被消元行只保存1元素的位置,且按升序排列,从而类似倒排链表数据结构。优点是存储空间占用更少,缺点是算法设计更复杂,并行化难度高。

\subsection{数据访问}

矩阵规模可能非常庞大,达到数百万行 / 列,难以全部放入内存。此时,需要设计的是外存算法,考虑如何分批次将数据读入内存进行处理,同时又保证正确性。对外存算法,算法分析和时间测试除了考虑计算之外,还要考虑 I / O 时间。

\subsection{并行化方法}

\begin{enumerate}
  \item 对位向量存储方式,两行间消元操作的并行化很直接,无论 SIMD、多线程还是 MPI、GPU,将向量拆分,子向量的异或即自然形成任务,可分配给不同的计算单元。当矩阵规模大到百万级别时,采取这种并行方式就够了。但当矩阵规模没有那么大时,一个消元操作计算量不足以支撑较大规模并行,就需要考虑消元操作间的并行,设计适合的并行任务划分,在提高并发度的同时保证正确性。
  \item 对类倒排链表存储方式,可以考虑循环展开和多线程优化。
  \item 当矩阵规模非常庞大,需使用外存算法时就要同时考虑计算和访存。多线程并行化时可考虑计算和访存异步模式,在前台线程进行消元计算时、后台线程读取下一步要处理的数据,这需要仔细设计计算和 I / O 步骤,降低依赖关系,以便实现异步模式。
\end{enumerate}

\section{研究计划}

\newpage

\section{参考文献}
\cite{1}\cite{2}

\bibliographystyle{plain}
\bibliography{Parallel-Programming-2.bib}

\end{document}