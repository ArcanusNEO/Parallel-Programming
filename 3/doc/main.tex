\documentclass[a4paper]{article}

\input{style/ch_xelatex.tex}
\input{style/scala.tex}

%代码段设置
\lstset{numbers=left,
basicstyle=\tiny,
numberstyle=\tiny,
keywordstyle=\color{blue!70},
commentstyle=\color{red!50!green!50!blue!50},
frame=single, rulesepcolor=\color{red!20!green!20!blue!20},
escapeinside=``
}

\graphicspath{ {figures/} }
\usepackage{ctex}
\usepackage{graphicx}
\usepackage{color,framed}%文本框
\usepackage{listings}
\usepackage{caption}
\usepackage{amssymb}
\usepackage{enumerate}
\usepackage{xcolor}
\usepackage{bm} 
\usepackage{lastpage}%获得总页数
\usepackage{fancyhdr}
\usepackage{tabularx}  
\usepackage{geometry}
\usepackage{minted}
\usepackage{graphics}
\usepackage{subfigure}
\usepackage{float}
\usepackage{pdfpages}
\usepackage{pgfplots}
\pgfplotsset{width=10cm,compat=1.9}
\usepackage{multirow}
\usepackage{footnote}
\usepackage{booktabs}

%-----------------------伪代码------------------
\usepackage{algorithm}  
\usepackage{algorithmicx}  
\usepackage{algpseudocode}  
\floatname{algorithm}{Algorithm}  
\renewcommand{\algorithmicrequire}{\textbf{Input:}}  
\renewcommand{\algorithmicensure}{\textbf{Output:}} 
\usepackage{lipsum}  
\makeatletter
\newenvironment{breakablealgorithm}
  {% \begin{breakablealgorithm}
  \begin{center}
     \refstepcounter{algorithm}% New algorithm
     \hrule height.8pt depth0pt \kern2pt% \@fs@pre for \@fs@ruled
     \renewcommand{\caption}[2][\relax]{% Make a new \caption
      {\raggedright\textbf{\ALG@name~\thealgorithm} ##2\par}%
      \ifx\relax##1\relax % #1 is \relax
         \addcontentsline{loa}{algorithm}{\protect\numberline{\thealgorithm}##2}%
      \else % #1 is not \relax
         \addcontentsline{loa}{algorithm}{\protect\numberline{\thealgorithm}##1}%
      \fi
      \kern2pt\hrule\kern2pt
     }
  }{% \end{breakablealgorithm}
     \kern2pt\hrule\relax% \@fs@post for \@fs@ruled
  \end{center}
  }
\makeatother
%------------------------代码-------------------
\usepackage{xcolor} 
\usepackage{listings} 
\lstset{ 
breaklines,%自动换行
basicstyle=\small,
escapeinside=``,
keywordstyle=\color{ blue!70} \bfseries,
commentstyle=\color{red!50!green!50!blue!50},% 
stringstyle=\ttfamily,% 
extendedchars=false,% 
linewidth=\textwidth,% 
numbers=left,% 
numberstyle=\tiny \color{blue!50},% 
frame=trbl% 
rulesepcolor= \color{ red!20!green!20!blue!20} 
}

%-------------------------页面边距--------------
\geometry{a4paper,left=2.3cm,right=2.3cm,top=2.7cm,bottom=2.7cm}
%-------------------------页眉页脚--------------
\usepackage{fancyhdr}
\pagestyle{fancy}
\lhead{\kaishu \leftmark}
% \chead{}
\rhead{\kaishu 并行程序设计实验报告}%加粗\bfseries 
\lfoot{}
\cfoot{\thepage}
\rfoot{}
\renewcommand{\headrulewidth}{0.1pt}  
\renewcommand{\footrulewidth}{0pt}%去掉横线
\newcommand{\HRule}{\rule{\linewidth}{0.5mm}}%标题横线
\newcommand{\HRulegrossa}{\rule{\linewidth}{1.2mm}}
\setlength{\textfloatsep}{10mm}%设置图片的前后间距
%--------------------文档内容--------------------

\begin{document}
\renewcommand{\contentsname}{目\ 录}
\renewcommand{\appendixname}{附录}
\renewcommand{\appendixpagename}{附录}
\renewcommand{\refname}{参考文献}
\renewcommand{\figurename}{图}
\renewcommand{\tablename}{表}
\renewcommand{\today}{\number\year 年 \number\month 月 \number\day 日}

%-------------------------封面----------------
\begin{titlepage}
  \begin{center}
    \includegraphics[width=0.8\textwidth]{NKU.png}\\[1cm]
    \vspace{20mm}
    \textbf{\huge\textbf{\kaishu{计算机学院}}}\\[0.5cm]
    \textbf{\huge{\kaishu{并行程序设计第 3 次作业}}}\\[2.3cm]
    \textbf{\Huge\textbf{\kaishu{高斯消去法的 SIMD 并行化}}}

    \vspace{\fill}

    % \textbf{\Large \textbf{并行程序设计期末实验报告}}\\[0.8cm]
    % \HRule \\[0.9cm]
    % \HRule \\[2.0cm]
    \centering
    \textsc{\LARGE \kaishu{姓名\ :\ 丁屹}}\\[0.5cm]
    \textsc{\LARGE \kaishu{学号\ :\ 2013280}}\\[0.5cm]
    \textsc{\LARGE \kaishu{专业\ :\ 计算机科学与技术}}\\[0.5cm]

    \vfill
    {\Large \today}
  \end{center}
\end{titlepage}

\renewcommand {\thefigure}{\thesection{}.\arabic{figure}}%图片按章标号
\renewcommand{\figurename}{图}
\renewcommand{\contentsname}{目录}
\cfoot{\thepage\ of \pageref{LastPage}}%当前页 of 总页数


% 生成目录
\clearpage
\tableofcontents
\newpage

\section{问题描述}
高斯消去的计算模式如图 \ref{gauss} 所示,在第$k$步时,对第$k$行从$(k, k)$开始进行除法操作,并且将后续的$k + 1$至$N$行进行减去第$k$行的操作,串行算法如下面伪代码所示。
\begin{figure}
  \centering
  \includegraphics[width=1\textwidth]{gauss.png}
  \caption{高斯消去法示意图}
  \label{gauss}
\end{figure}
\begin{breakablealgorithm}
  \caption{普通高斯消元算法伪代码}
  \begin{algorithmic}[1] %每行显示行号  
    \Function {LU}{}
    \For {$k:=0$\ \textbf{to}\ $n$}
    \For {$j:=k+1$\ \textbf{to}\ $n$}
    \State {$A[k,j]:=A[k,j]/A[k,k]$}
    \EndFor
    \State{$A[k,k]:=1.0$}
    \For {$i:=k+1$\ \textbf{to}\ $n$}
    \For {$j:=k+1$\ \textbf{to}\ $n$}
    \State {$A[i,j]:=A[i,j]-A[i,k]*A[k,j]$}
    \EndFor
    \State{$A[i,k]:=0$}
    \EndFor
    \EndFor
    \EndFunction
  \end{algorithmic}
\end{breakablealgorithm}

观察高斯消去算法,注意到伪代码第 4,5 行第一个内嵌循环中的$A[k, j] := A[k, j]/A[k, k]$以及伪代码第$8,9,10$行双层$for$循环中的$A[i, j] := A[i, j] − A[i, k]×A[k, j]$都是可以进行向量化的循环。可以通过SIMD 扩展指令对这两步进行并行优化。

\section{SIMD算法设计}

源码链接:
\url{https://github.com/ArcanusNEO/Parallel-Programming/tree/master/3}

\subsection{测试用例的确定}

由于测试数据集较大,不便于各个平台同步,所以采用固定随机数种子为 $12345687$ 的 mt19937随机数生成器。
经过实验发现不同规模下,所有元素独立生成,限制大小在 $[0, 100]$,能够生成可以被正确消元的矩阵。

代码如下:

\begin{lstlisting}[title=测试数据集生成器,frame=trbl,language={C++}]
  uniform_real_distribution<float> dist(0, 100);
  mt19937       mt(12345687);
  int           n;
  istringstream iss(argv[1]);
  iss >> n;
  cout << n << endl;
  for (int i = 1; i <= n; ++i)
    for (int j = 1; j <= n; ++j) cout << dist(mt) << " \n"[j == n];
\end{lstlisting}

\subsection{算法设计}

\subsubsection{默认平凡算法}

编译选项:-O3;

使用一维数组模拟矩阵,避免改变矩阵大小时第二维不方便调整、必须设成最大值的问题,可以减少 cache 失效;

使用 $\#define\ matrix(i, j)\ arr[i * n + j]$ 宏,增强可读性;

\begin{lstlisting}[title=平凡算法,frame=trbl,language={C++}]
void func(int& ans, float arr[], int n) {
#define matrix(i, j) arr[i * n + j]
  for (int k = 0; k < n; ++k) {
    for (int j = k + 1; j < n; ++j) matrix(k, j) = matrix(k, j) / matrix(k, k);
    matrix(k, k) = 1.0;
    for (int i = k + 1; i < n; ++i) {
      for (int j = k + 1; j < n; ++j)
        matrix(i, j) = matrix(i, j) - matrix(i, k) * matrix(k, j);
      matrix(i, k) = 0;
    }
  }
#undef matrix
}
\end{lstlisting}

\subsubsection{使用 NEON 指令集并行化加速}

实验在华为鲲鹏 ARM 集群平台完成;

编译选项:-march=native 或 -march=armv8-a,全部开启 -O3;

\begin{lstlisting}[title=NEON 内存未对齐算法,frame=trbl,language={C++}]
void func(int& ans, float arr[], int n) {
#define matrix(i, j)  arr[i * n + j]
#define pmatrix(i, j) (arr + (i * n + j))
  for (int k = 0; k < n; ++k) {
    auto vt = vdupq_n_f32(matrix(k, k));
    int  j;
    for (j = k + 1; j + 4 <= n; j += 4) {
      auto va = vld1q_f32(pmatrix(k, j));
      va      = vdivq_f32(va, vt);
      vst1q_f32(pmatrix(k, j), va);
    }
    for (; j < n; ++j) matrix(k, j) = matrix(k, j) / matrix(k, k);
    matrix(k, k) = 1.0;
    for (int i = k + 1; i < n; ++i) {
      auto vaik = vdupq_n_f32(matrix(i, k));
      for (j = k + 1; j + 4 <= n; j += 4) {
        auto vakj = vld1q_f32(pmatrix(k, j));
        auto vaij = vld1q_f32(pmatrix(i, j));
        auto vx   = vmulq_f32(vakj, vaik);
        vaij      = vsubq_f32(vaij, vx);
        vst1q_f32(pmatrix(i, j), vaij);
      }
      for (; j < n; ++j)
        matrix(i, j) = matrix(i, j) - matrix(i, k) * matrix(k, j);
      matrix(i, k) = 0;
    }
  }
#undef matrix
#undef pmatrix
}
\end{lstlisting}

\begin{lstlisting}[title=NEON 内存对齐算法,frame=trbl,language={C++}]
void func(int& ans, float arr[], int n) {
#define matrix(i, j)  arr[i * n + j]
#define pmatrix(i, j) (arr + (i * n + j))
  for (int k = 0; k < n; ++k) {
    auto vt = vdupq_n_f32(matrix(k, k));
    int  j  = k + 1;
    for (; j < n && j % 4; ++j) matrix(k, j) = matrix(k, j) / matrix(k, k);
    for (; j + 4 <= n; j += 4) {
      auto va = vld1q_f32(pmatrix(k, j));
      va      = vdivq_f32(va, vt);
      vst1q_f32(pmatrix(k, j), va);
    }
    for (; j < n; ++j) matrix(k, j) = matrix(k, j) / matrix(k, k);
    matrix(k, k) = 1.0;
    for (int i = k + 1; i < n; ++i) {
      auto vaik = vdupq_n_f32(matrix(i, k));
      for (j = k + 1; j < n && j % 4; ++j)
        matrix(i, j) = matrix(i, j) - matrix(i, k) * matrix(k, j);
      for (; j + 4 <= n; j += 4) {
        auto vakj = vld1q_f32(pmatrix(k, j));
        auto vaij = vld1q_f32(pmatrix(i, j));
        auto vx   = vmulq_f32(vakj, vaik);
        vaij      = vsubq_f32(vaij, vx);
        vst1q_f32(pmatrix(i, j), vaij);
      }
      for (; j < n; ++j)
        matrix(i, j) = matrix(i, j) - matrix(i, k) * matrix(k, j);
      matrix(i, k) = 0;
    }
  }
#undef matrix
#undef pmatrix
}
\end{lstlisting}

可以看到内存对齐算法就是在 $j := k + 1$ 之后先平凡地计算,直到 $j \% 4 = 0$,保证了偏移 $4 × 4 = 16$ 字节对齐 由于使用了 $aligned\_alloc$ 分配了 32 字节对齐的内存块,故保证了 16 字节对齐。

\subsubsection{使用 SSE、AVX 指令集并行化加速}

实验在本地 x86 平台完成;

编译选项:-march=native -O3;

x86 下的 SSE、AVX 指令与 NEON 指令有相似的定义,可以方便地转换。内存对齐方式也与 NEON 的内存对齐方式类似,篇幅原因不再给出代码。

\begin{lstlisting}[title=SSE 内存未对齐算法,frame=trbl,language={C++}]
void func(int& ans, float arr[], int n) {
#define matrix(i, j)  arr[i * n + j]
#define pmatrix(i, j) (arr + (i * n + j))
  for (int k = 0; k < n; ++k) {
    auto vt =
      _mm_set_ps(matrix(k, k), matrix(k, k), matrix(k, k), matrix(k, k));
    int j;
    for (j = k + 1; j + 4 <= n; j += 4) {
      auto va = _mm_loadu_ps(pmatrix(k, j));
      va      = _mm_div_ps(va, vt);
      _mm_storeu_ps(pmatrix(k, j), va);
    }
    for (; j < n; ++j) matrix(k, j) = matrix(k, j) / matrix(k, k);
    matrix(k, k) = 1.0;
    for (int i = k + 1; i < n; ++i) {
      auto vaik =
        _mm_set_ps(matrix(i, k), matrix(i, k), matrix(i, k), matrix(i, k));
      for (j = k + 1; j + 4 <= n; j += 4) {
        auto vakj = _mm_loadu_ps(pmatrix(k, j));
        auto vaij = _mm_loadu_ps(pmatrix(i, j));
        auto vx   = _mm_mul_ps(vakj, vaik);
        vaij      = _mm_sub_ps(vaij, vx);
        _mm_storeu_ps(pmatrix(i, j), vaij);
      }
      for (; j < n; ++j)
        matrix(i, j) = matrix(i, j) - matrix(i, k) * matrix(k, j);
      matrix(i, k) = 0;
    }
  }
#undef matrix
#undef pmatrix
}
\end{lstlisting}

\begin{lstlisting}[title=AVX 内存未对齐算法,frame=trbl,language={C++}]
void func(int& ans, float arr[], int n) {
#define matrix(i, j)  arr[i * n + j]
#define pmatrix(i, j) (arr + (i * n + j))
  for (int k = 0; k < n; ++k) {
    auto vt =
      _mm256_set_ps(matrix(k, k), matrix(k, k), matrix(k, k), matrix(k, k),
                    matrix(k, k), matrix(k, k), matrix(k, k), matrix(k, k));
    int j;
    for (j = k + 1; j + 8 <= n; j += 8) {
      auto va = _mm256_loadu_ps(pmatrix(k, j));
      va      = _mm256_div_ps(va, vt);
      _mm256_storeu_ps(pmatrix(k, j), va);
    }
    for (; j < n; ++j) matrix(k, j) = matrix(k, j) / matrix(k, k);
    matrix(k, k) = 1.0;
    for (int i = k + 1; i < n; ++i) {
      auto vaik =
        _mm256_set_ps(matrix(i, k), matrix(i, k), matrix(i, k), matrix(i, k),
                      matrix(i, k), matrix(i, k), matrix(i, k), matrix(i, k));
      for (j = k + 1; j + 8 <= n; j += 8) {
        auto vakj = _mm256_loadu_ps(pmatrix(k, j));
        auto vaij = _mm256_loadu_ps(pmatrix(i, j));
        auto vx   = _mm256_mul_ps(vakj, vaik);
        vaij      = _mm256_sub_ps(vaij, vx);
        _mm256_storeu_ps(pmatrix(i, j), vaij);
      }
      for (; j < n; ++j)
        matrix(i, j) = matrix(i, j) - matrix(i, k) * matrix(k, j);
      matrix(i, k) = 0;
    }
  }
#undef matrix
#undef pmatrix
}
\end{lstlisting}

\section{实验及结果分析}
% Please add the following required packages to your document preamble:
% \usepackage{graphicx}
\begin{table}[]
  \centering
  \resizebox{\textwidth}{!}{%
    \begin{tabular}{llllllllll}
      Scale       & Reperat times & x86 ordinary (s) & x86 SSE unaligned (s) & x86 SSE aligned (s) & x86 AVX unaligned (s) & x86 AVX aligned (s) & arm ordinary (s) & arm NEON unaligned (s) & arm NEON aligned (s) \\
      8 × 8       & 100           & 0.000001330460   & 0.000001379360        & 0.000001326980      & 0.000001479190        & 0.000001370300      & 0.000000525400   & 0.000000541900         & 0.000000484900       \\
      16 × 16     & 50            & 0.000001706920   & 0.000001955580        & 0.000001729300      & 0.000001857780        & 0.000001653840      & 0.000001666000   & 0.000001702000         & 0.000001401600       \\
      32 × 32     & 50            & 0.000003640080   & 0.000004258960        & 0.000003834360      & 0.000004063340        & 0.000003761560      & 0.000007127000   & 0.000008225000         & 0.000006609400       \\
      64 × 64     & 20            & 0.000015253300   & 0.000021343650        & 0.000015731900      & 0.000015885350        & 0.000014212650      & 0.000037566500   & 0.000051374500         & 0.000040106500       \\
      128 × 128   & 15            & 0.000098880800   & 0.000121785400        & 0.000096791267      & 0.000089601000        & 0.000077025000      & 0.000231574000   & 0.000349447333         & 0.000275770000       \\
      256 × 256   & 10            & 0.000716408500   & 0.000878330700        & 0.000622779500      & 0.000588631100        & 0.000452169800      & 0.001820356000   & 0.002725389000         & 0.002021820000       \\
      512 × 512   & 10            & 0.006722607300   & 0.007444306600        & 0.005199414400      & 0.005004266800        & 0.003661343100      & 0.014974396000   & 0.021199319000         & 0.016449107000       \\
      1024 × 1024 & 5             & 0.064893815400   & 0.084234089600        & 0.063566091600      & 0.052573721800        & 0.047696106600      & 0.135511226000   & 0.196087378000         & 0.137847262000       \\
      2048 × 2048 & 3             & 1.400074583333   & 1.534334469667        & 1.501020112000      & 1.366758369000        & 1.317500752000      & 1.101775523333   & 1.511987780000         & 1.093070066667       \\
      4096 × 4096 & 1             & 10.705585484000  & 11.438922085000       & 11.608043578000     & 9.813709308000        & 10.472416784000     & 13.088073440000  & 14.938220180000        & 14.540709080000
    \end{tabular}%
  }
  \caption{所有平台结果对比}
  \label{tab:result}
\end{table}

从表 \ref{tab:result} 中可以发现:
\begin{itemize}
  \item x86 平台比 arm 平台普遍更快,可能是 CPU 单核性能差异的结果
  \item 单纯使用 SIMD 指令但不进行内存对齐带来的性能优化有限,甚至比 O3 优化的平凡算法慢
  \item 使用 SIMD 指令且进行内存对齐可以带来明显的性能提升
  \item SSE 的指令级并行程度不够,提升不太明显,但内存对齐的 AVX 指令性能优化明显,某些情况下对比平凡算法速度有近一倍的提升
  \item 在 arm 中,指令级并行加速效果不明显,甚至比 O3 优化的平凡算法慢,内存对齐的 NEON 指令在某些情况下可以超过 O3 优化的平凡算法
\end{itemize}
% \newpage

% \section{参考文献}
% \cite{1}\cite{2}\cite{3}\cite{4}\cite{5}\cite{6}

% \bibliographystyle{plain}
% \bibliography{Parallel-Programming-0.bib}

\end{document}